\documentclass{beamer}
\usetheme{metropolis}

\begin{document}

\title{Répartition de ressources par homologie persistante : 
    application au développement des transports publics}
\author{Harnisch Elowan}
\date{\today}

\maketitle

\begin{frame}
    \frametitle{Présentation de l'homologie persistante}
    Permet de mesurer et d'évaluer comment les ressources sont réparties 
    géographiquement. En particulier, elle permet de détecter les trous
    géographiques dans les villes, de voir où il y a des zones non desservies,
    des clusters de zones desservies par les transports publics.

\end{frame}

\begin{frame}
    \frametitle{Explication de l'approche}
    \itemize{
        \item Construction d'un complexe simplicial à partir des données géographiques
        \item Utilisation d'une distance basée sur le temps de trajet et l'affluence
        \item Utilisation de la persistance homologique pour mesurer la couverture
    }
\end{frame}

\begin{frame}
    \frametitle{Récupération des données}
    Todo
\end{frame}

\begin{frame}
    \frametitle{Plus de détails sur la l'homologie persistante}
    information de la date de naissance et de mort des trous dans la couverture
    des ressources.
    
\end{frame}

\begin{frame}
    \frametitle{La construction du "graphe"}
    Todo
\end{frame}

\begin{frame}
    \frametitle{Résultats}
    Todo
\end{frame}

\begin{frame}
    \frametitle{Conclusion}
    ça marche
\end{frame}

\end{document}