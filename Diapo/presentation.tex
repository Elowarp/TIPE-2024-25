\documentclass{beamer}

\usetheme{metropolis}

\usepackage{minted}
\title{Minimisation de machines de Turing, application au problème 2Color}
\date{\today}
\author{Elowan Harnisch}
\begin{document}
    \maketitle
    \begin{frame}{Machine de Turing}
        \begin{alert}{Définition}
            Une machine de Turing est un sixtuplet $(Q, \Sigma, S, q_0, F, \delta)$ où:
            \begin{itemize}
                \item $Q$ est un ensemble fini d'états
                \item $\Sigma$ est un ensemble fini de symboles
                \item $S \notin \Sigma$ est le symbole blanc
                \item $q_0 \in Q$ est l'état initial
                \item $F \subseteq Q$ est l'ensemble des états acceptants
                \item $\delta: Q \times \Sigma \rightarrow Q \times \Sigma \times \{L, R\}$ est la fonction de 
                transition
            \end{itemize}
            Elle possède de plus une bande infinie sur laquelle elle peut
            lire et écrire les symboles de $\Sigma \cup {S}$. Elle sert de 
            modèle théorique de l'ordinateur.
        \end{alert}
    \end{frame}
    \begin{frame}{Machine de Turing}
        \begin{figure}
            \includegraphics[width=0.5\textwidth]{images/turing_machine.png}
            Schema d'execution de machine de Turing
        \end{figure}
    \end{frame}
    \begin{frame}{Automate fini déterministe (DFA)}
        \begin{alert}{Définition}
            Un automate fini déterministe (DFA) est un quintuplet 
            $(Q, \Sigma, q_0, F, \delta)$ où:
            \begin{itemize}
                \item $Q$ est un ensemble fini d'états
                \item $\Sigma$ est un ensemble fini de symboles
                \item $q_0 \in Q$ est l'état initial
                \item $F \subseteq Q$ est l'ensemble des états finaux
                \item $\delta: Q \times \Sigma \rightarrow Q$ est la 
                fonction de transition
            \end{itemize}
            C'est un modèle qui permet grâce à une construction en graphe de 
            reconnaitre l'appartenance de mots à un langage. 
        \end{alert}
    \end{frame}
    \begin{frame}{Automate fini déterministe (DFA)}
        \begin{figure}
            \includegraphics[width=0.5\textwidth]{images/turing_machine.png}
            Schema d'un automate reconnaissant le langage des mots de longueur 
            paire
        \end{figure}
    \end{frame}
    \begin{frame}{Minimisation d'un DFA}
        \begin{alert}{Définition}
            La minimisation d'un automate fini déterministe est le processus 
            qui consiste à réduire le nombre d'états de l'automate tout en 
            conservant le langage reconnu. 
        \end{alert}
        \begin{block}{Théorème}
            L'algorithme de Hopcroft permet depuis un automate fini déterministe 
            d'obtenir un automate minimal reconnaissant le même langage.
        \end{block}
    \end{frame}
    \begin{frame}{Minimisation d'un automate fini déterministe}
        % \begin{minted}{python}
            Code de l'algorithme de Hopcroft
        % \end{minted}
    \end{frame}
    \begin{frame}{Minimisation d'une machine de Turing}
        \begin{alert}{Transformation d'une machine de Turing en DFA}
            On en déduit son inverse trivialement. Celle ci ne forme en revanche pas 
            une bijection.
        \end{alert}
        \begin{block}{Théorème}
            La composition de la transformation d'une machine de Turing en DFA, de 
            l'algorithme de minimisation de Hopcroft et de la transformation inverse
            permet d'obtenir une machine de Turing de taille au plus égale à celle de
            départ.
        \end{block}
    \end{frame}
    \begin{frame}{Problème 2Color}
        \begin{alert}{Définition}
            Le problème 2Color est un problème de décision qui consiste à déterminer 
            si un graphe est coloriable avec 2 couleurs. 
        \end{alert}
        \begin{block}{Théorème}
            Le problème 2Color est décidable, donc il existe une machine de turing qui 
            le calcule.
        \end{block}
    \end{frame}
    \begin{frame}{Problème 2Color}
        \begin{figure}
            \includegraphics[width=0.5\textwidth]{images/2color.png}
            Machine de Turing créee pour résoudre le problème 2Color
        \end{figure}
    \end{frame}
    \begin{frame}{Conclusion}
        Pour la suite j'aimerais :
        \begin{itemize}
            \item Bien comprendre la preuve de l'algorithme de Hopcroft et mieux 
            l'adapter aux machines de Turing plutôt que mon bricolage actuel (même 
            si je pense que ce n'est pas possible, dans ce cas démontrer que ce problème
            n'est pas décidable mais qu'on peut essayer de s'en approcher par cette méthode)
            \item Finir de créer la machine de Turing pour le problème 2Color 
        \end{itemize}
    \end{frame}
\end{document}