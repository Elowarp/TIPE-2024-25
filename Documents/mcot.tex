\documentclass{article}

\begin{document}

\title{Optimisation de machines de Turing, application à une version du problème 2Color}
\author{Elowan}
\date{\today}

\maketitle

\section{Introduction}
Ce TIPE tente de s'atteler à la minimisation de machines de Turing, en 
s'inspirant de la minimisation d'automates finis déterministes. Essayant
de trouver une procédure afin de réduire une machine de Turing à un automate
fini déterministe, à appliquer l'algorithme de Hopcroft puis à reconstruire
la machine de Turing finale. Enfin, on appliquera cette méthode à une machine
de Turing qui résout un problème de coloriage de graphe. En l'occurrence,
le problème n-2Color, qui vise à déterminer si un graphe de maximum n 
sommets est coloriable avec deux couleurs. 

\section{Bibliographie commentée}
L'étude de la minimisation de machines de Turing est un sujet peu abordé
malgré son intérêt théorique. Nous pouvons citer l'intelligence artificelle 
pour laquelle la minimisation de machines de Turing pourrait être utile.
De plus, la théorie des automates est un sujet très riche, qui a été étudié
par de nombreux mathématiciens et informaticiens depuis les années 50. Notamment
avec les travaux de Hopcroft, qui a donné un algorithme de minimisation 
d'automates finis déterministes [1-2]. 

Les machines de Turing ainsi que les automates étant des modèles de calculs 
proches, nous nous proposons d'étudier une transformation d'une machine de 
Turing en automate fini déterministe, qui nous permettra d'appliquer 
l'algorithme de Hopcroft. Cette transformation ne nous permet pas d'utiliser 
l'automate comme une machine de Turing, mais nous offre une bijection entre 
les machines de turing et les automates obtenus par cette transformation. 

Une fois cette procédure établie, nous pourrons l'appliquer à un problème de
coloriage de graphe, le problème n-2Color. Ce problème est un problème de
décision qui consiste à déterminer si un graphe est coloriable avec deux
couleurs pour un graphe de maximum n sommets. Ce problème est décidable étant 
décidable, nous savons qu'il existe une machine de Turing qui le modélise [Source?]. 
Ainsi, dans ce TIPE, nous développerons une machine de Turing qui résout ce
problème, puis nous cherchons à la minimiser en utilisant la procédure définie
précédémment, afin d'attester de la validité de cette méthode.

\section{Problématique retenue}
Comment obtenir une version plus petite d'une machine de Turing, tout en
conservant le même langage ? Et comment appliquer cette méthode à un problème
de coloriage de graphe ?

\section{Objectifs du TIPE du candidat}
\begin{itemize}
    \item Développer une procédure de réduction d'états d'une machine de Turing
    \item Construire une machine de Turing qui résout le problème n-2Color
\end{itemize}

\section{Références bibliographiques}
\begin{enumerate}
    \item Beauquier D, Berstel J : Eléments d'algorithmique (2005)
    \item Carton O : Langages formels, Calculabilité et complexité (2014)
\end{enumerate}

\end{document}